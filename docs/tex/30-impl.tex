\chapter{Технологическая часть}

В данном разделе представленны средства разработки программного обеспечения, детали реализации и тестирование функций.

\section{Средства реализации}

В качестве языка программирования, на котором будет реализовано программное обеспечение, выбран язык программирования JavaScript \cite{js}. Выбор языка обусловлен тем, что для этого языка существует библиотека QCEngine \cite{qcengine}, предоставляющая полный функционал который нужен для реализации выбранного мною алгоритма. В качестве программной платформы, с помощью которого можно превратить JavaScript из узкоспециализированного языка (работающего только в браузере) в язык общего назначения, был выбран Node.js \cite{nodejs}, так как он на данный момент никаких альтернатив ему нет.

Для создания пользовательского интерфейса программного обеспечения будет использоваться модуль node-gtk \cite{node-gtk} вместе с модулем canvas \cite{node-canvas}. Два этих инструмента в связке друг с другом позволят одновременно выводить синтезируемое изображение на экран, и сохранять его (при необходимости) в файл, например для дальнейшего анализа с помощью каких-либо других утилит.

Для тестирования программного обеспечения будет использоваться фреймворк Jest \cite{jest}. Данный инструмент предоставляет широкие возможности для тестирования приложений, в том числе написанных на Node.js.

Для обеспечения качества кода был использован инструмент ESLint \cite{eslint}, позволяющий во время процесса написания исходных кодов программного обеспечения контролировать наличие синтаксических и логических ошибок.

В качестве среды разработки выбран текстовый редактор Visual Studio Code \cite{vscode}, содержащий большим количеством плагинов и инструментов для различных языков программирования, в том числе JavaScript. Такие инструменты облегчают и ускоряют процесс разработки программного обеспечения.

\section{Детали реализации}

В листингах 3.1 -- 3.3 приведен исходный код реализации алгоритма квантовой супер выборки. Сам алгоритм разделен на подпрограммы: формирование квантовой поисковой таблицы, синтез изображения и заполнения карты достоверности. 


\begin{lstlisting}[label=qtable,caption=Функция формирования квантовой поисковой таблицы, language=javascript]
jabascript
\end{lstlisting}

\begin{lstlisting}[label=synt,caption=Функция синтеза изображения на основе значений поисковой таблицы, language=javascript]
javascript2
\end{lstlisting}

\begin{lstlisting}[label=qfind,caption=Функция заполнения карты достоверности, language=javascript]
javascript3
\end{lstlisting}

\section*{Вывод}

В данном разделе были рассмотренны средства реализации программного обеспечения и листинги исходных кодов программного обеспечения, разработанного на основе алгоритма, изложенного в конструкторской части.
