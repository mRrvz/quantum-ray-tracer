\chapter{Технологическая часть}

В данном разделе представленны средства разработки программного обеспечения, детали реализации и тестирование функций.

\section{Средства реализации}

В качестве языка программирования, на котором будет реализовано программное обеспечение, выбран язык программирования JavaScript \cite{js}. Выбор языка обусловлен тем, что для этого языка существует библиотека QCEngine \cite{qcengine}, предоставляющая полный функционал который нужен для реализации выбранного мною алгоритма. В качестве программной платформы, с помощью которого можно превратить JavaScript из узкоспециализированного языка (работающего только в браузере) в язык общего назначения, был выбран Node.js \cite{nodejs}.

Для создания пользовательского интерфейса программного обеспечения будет использоваться модуль node-gtk \cite{node-gtk} вместе с модулем canvas \cite{node-canvas}. Два этих инструмента в связке друг с другом позволят одновременно выводить синтезируемое изображение на экран, и сохранять его (при необходимости) в файл, например для дальнейшего анализа с помощью каких-либо других утилит.

Для тестирования программного обеспечения будет использоваться фреймворк Jest \cite{jest}. Данный инструмент предоставляет много различного функционала, позволяющего тестировать приложения, в том числе написанных на Node.js. Разрабатываемое программное обеспечение будет тестироваться по средствам модульного тестирования \cite{utests}. Функциональное тестирование \cite{ftests} ПО проводиться не будет из-за своей специфики -- разработанное ПО является GUI-приложением, что усложняет процесс тестирования.

Для обеспечения качества кода был использован инструмент ESLint \cite{eslint}, позволяющий во время процесса написания исходных кодов программного обеспечения контролировать наличие синтаксических и логических ошибок.

В качестве среды разработки выбран текстовый редактор Visual Studio Code \cite{vscode}, содержащий большим количеством плагинов и инструментов для различных языков программирования, в том числе JavaScript. Такие инструменты облегчают и ускоряют процесс разработки программного обеспечения \cite{why-ide}.

\section{Детали реализации}

В листингах \ref{qtable} -- \ref{qmap} приведен исходный код реализации алгоритма квантовой супер выборки. Сам алгоритм разделен на подпрограммы: формирование квантовой поисковой таблицы, синтез изображения и заполнения карты достоверности. 

\begin{lstlisting}[label=qtable,caption=Функция формирования квантовой поисковой таблицы, language=javascript]
function create_qss_lookup_table()
{
	qc.reset((Constants.res_aa_bits + Constants.res_aa_bits) + Constants.num_counter_bits);
	var qxy = qc.new_qint(Constants.res_aa_bits + Constants.res_aa_bits, 'qxy');
	var qcount = qc.new_qint(Constants.num_counter_bits, 'count');
	
	var num_subpixels = 1 << (Constants.res_aa_bits + Constants.res_aa_bits)
	qss_full_lookup_table = null;
	
	for (var hits = 0; hits <= num_subpixels; ++hits)
	{
		create_table_column(hits, qxy, qcount);
	}
	
	var cw = qss_full_lookup_table;
	qss_count_to_hits = [];
	
	for (var count = 0; count < cw.length; ++count)
	{
		var best_hits = 0;
		var best_prob = 0;
	
		for (var hits = 0; hits < cw[0].length; ++hits)
		{
			if (best_prob < cw[count][hits])
			{
				best_prob = cw[count][hits];
				best_hits = hits;
			}
		}
	
		qss_count_to_hits.push(best_hits);
	}
	
	if (qss_full_lookup_table && images.display_cwtable)
	{
		images.display_cwtable.setup(cw[0].length, cw.length, 16);
		
		for (var y = 0; y < cw.length; ++y)
		{
			for (var x = 0; x < cw[0].length; ++x)
			{
				images.display_cwtable.pixel(x, y, cw[y][x]);
			}
		}
	}
}


function create_table_column(color, qxy, qcount)
{
	var num_subpixels = 1 << (Constants.res_aa_bits + Constants.res_aa_bits)
	var true_count = color;
	
	qc.write(0);
	qcount.hadamard();
	qxy.hadamard();
	
	for (var i = 0; i < Constants.num_counter_bits; ++i)
	{
		var reps = 1 << i;
		var condition = qcount.bits(reps);
		var mask_with_condition = qxy.bits().or(condition);
	
		for (var j = 0; j < reps; ++j)
		{
			flip_n_terms(qxy, true_count, condition);
			grover_iteration(qxy.bits(), mask_with_condition);
		}
	}
	
	invQFT(qcount);
	var table = [];
	
	for (var i = 0; i < (1 << Constants.num_counter_bits); ++i)
	{
		table.push(qcount.peekProbability(i));
	}
	
	if (qss_full_lookup_table == null)
	{
		qss_full_lookup_table = [];
	
		for (var i = 0; i < (1 << Constants.num_counter_bits); ++i)
		{
			qss_full_lookup_table.push([]);
			
			for (var j = 0; j < num_subpixels; ++j)
			{
				qss_full_lookup_table[i].push(0);
			}
		}
	}
	
	for (var col = 0; col < 1 << Constants.num_counter_bits; ++col)
	{
		qss_full_lookup_table[col][true_count] = table[col];
	}
}
\end{lstlisting}

\begin{lstlisting}[label=qss,caption=Функция синтеза изображения на основе значений поисковой таблицы, language=javascript]
function do_qss_image()
{
	qc.reset(2 * Constants.res_aa_bits + Constants.num_counter_bits + Constants.accum_bits);
	var sp = {};
	
	sp.qx = qc.new_qint(Constants.res_aa_bits, 'qx');
	sp.qy = qc.new_qint(Constants.res_aa_bits, 'qy');
	sp.counter = qc.new_qint(Constants.num_counter_bits, 'counter');
	sp.qacc = qc.new_qint(Constants.accum_bits, 'scratch');
	sp.qacc.write(0);
	
	var total_pixel_error = 0;
	var num_zero_error_pixels = 0;
	qss_raw_result = [];
	
	for (sp.ty = 0; sp.ty < Constants.res_tiles; ++sp.ty)
	{
		qss_raw_result.push([]);
		
		Constants.update_fraction();
		
		for (sp.tx = 0; sp.tx < Constants.res_tiles; ++sp.tx)
		{
			qss_tile(sp);
			qss_raw_result[sp.ty].push(sp.readVal);
			images.display_qss.pixel(sp.tx, sp.ty, sp.color);
			
			if (ideal_result)
			{
				var pixel_error = Math.abs(sp.hits - ideal_result[sp.ty][sp.tx]);
			
				if (pixel_error)
				{
					total_pixel_error += pixel_error;
				}
				else
				{
					num_zero_error_pixels++;
				}
			}
		}
	}
}
\end{lstlisting}

\begin{lstlisting}[label=qmap,caption=Функция заполнения карты достоверности, language=javascript]
function draw_confidence_map()
{
	var cw = qss_full_lookup_table;
	for (var ty = 0; ty < Constants.res_tiles; ++ty)
	{
		for (var tx = 0; tx < Constants.res_tiles; ++tx)
		{
			var qss_out = qss_raw_result[ty][tx];
			var row_total = 0;
			var row_max = 0;
		
			for (var x = 0; x < cw[0].length; ++x)
			{
				row_total += cw[qss_out][x];
					
				if (cw[qss_out][x] > row_max)
				{
					row_max = cw[qss_out][x];
				}
			}
	
			images.display_confidence.pixel(tx, ty, row_max / row_total);
		}
	}
}
\end{lstlisting}

\section*{Вывод}

В данном разделе были рассмотренны средства реализации программного обеспечения и листинги исходных кодов программного обеспечения, разработанного на основе алгоритма, изложенного в конструкторской части.
