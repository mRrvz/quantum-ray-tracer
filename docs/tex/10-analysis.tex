\chapter{Аналитическая часть}

\section{Основы квантовых вычислений}

\subsection{Квантовый бит}

В квантовых вычислениях физические свойства квантовых объектов реализованы в кубитах. Классический бит принимает только два значения – 0 или 1. Кубит до измерения принимает одновременно оба значения. Из-за этого, кубит принято обозначать выражением $a\ket{0}$ + $b\ket{1}$, где $\alpha$ и $\beta$ — комплексные числа, удовлетворяющие условию (\ref{for:ver})

\begin{equation}
\label{for:ver}
|\alpha|^2 + |\beta|^2 = 1. 
\end{equation} 

Измерение кубита мгновенно «схлопывает»  его состояние в базисное – 0 или 1. Вероятности перехода в эти состояние равны соотвественно $|\alpha|^2$ и $|\beta|^2$. 

\subsection{Квантовый регистр}

На кубиты может быть наложена ненаблюдаемая связь -- при всяком изменении над одним из нескольких кубитов остальные меняются согласованно с ним. Таким образом, можно интерпертировать такую совокупность кубитов как заполненный квантовый регистр. Такой регистр может находиться во всех комбинациях составляющих его битов, и, кроме этого, реализовывать завсимости между ними.

\section{Синтез изображения в квантовом представлении}

\subsection{Квантовый пиксельный шейдер}

Пиксельный шейдер -- это программа (которая чаще всего выполняется на графическом процессоре), которая на вход принимает координаты $x$ и $y$ и на выходе выдает цвет пикселя, находящегося в заднных координатах. Для реализации квантового пиксельного шейдера, воспользуемся двумя квантовыми регистрами, каждый из которых состоит из $N$ кубитов. Назовём их $qx$ и $qy$ для осей $x$ и $y$ соответсвенно. Таким образом, размер синтезируемоего изображения зависит от величины $N$. Например, при $N = 4$, регистры $qx$ и $qy$ будут содержать $2^N = 2^4 = 16$ кубитов, что будет соответствовать размеру изображения 16x16 пикселей. Такой пиксель может быть только черным (1) или белым (0), потому что он будет соответсвовать значениям которые может принимать один кубит. 

\subsection{Квантовая фаза}

Произвольное квантовое состояние, обозначаемое $\ket{\psi}$, может быть любая суперпозиция, записанная в виде (\ref{for:psi}):
\begin{equation}
\label{for:psi}
\ket{\psi} = \alpha\ket{0} + \beta\ket{1}
\end{equation}
 базисных векторов. Согласно условию (\ref{for:ver}) и основываясь на том факте, что глобальная фаза не наблюдаема (то есть $\ket{\psi}$ тоже самое что и $e^{j\gamma}\ket{\psi}$) \cite{global-phase}, выражение (\ref{for:psi}) можно переписать в виде (\ref{for:new_ver}):
 
\begin{equation} 
\label{for:new_ver}
\ket{\psi} = \sqrt{1 - p}\ket{0} + e^{j\gamma}\sqrt{p}\ket{1}
\end{equation} где $0 \leq p \leq 1$ -- вероятность того, что бит находится в состоянии 1 и $0 \leq \psi < 2\pi$ -- квантовая фаза.

Применив операцию смены фазы на противоположную (повернуть кубит на 180 градусов), можно изменить состояние кубита аналогично на противоположное. Операцию смены фазы можно применять сразу на несколько кубитов. Таким образом, всего за одну операцию, можно, например, изменить цвет половины (или всех) пикселей синтезируемого изображения. Выигрыш по сравнению с обычными вычислениями очевиден. Кроме того, стало понятно, как можно изобразить прямые. 

\subsection{Логические операции в квантовой фазе}

Chapter 10 

Окружность задается уравнением вида (\ref{for:circle}):

\begin{equation}
\label{for:circle}
x^2 + y^2 = r^2
\end{equation}

Нам известно, как можно сохранить значение логической операции в фазе. Таким образом, стало понятно, как можно изобразить окружность, ее четверть или более сложные кривые.

\section{Квантовое суперсэмплирование}
\section{Колоризация изображения}

\section*{Вывод}

