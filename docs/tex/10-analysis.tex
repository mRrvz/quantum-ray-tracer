\chapter{Аналитическая часть}

\section{Основы квантовых вычислений}

\subsection{Квантовый бит}

В квантовых вычислениях физические свойства квантовых объектов реализованы в кубитах. Классический бит принимает только два значения – 0 или 1. Кубит до измерения принимает одновременно оба значения. Из-за этого, кубит принято обозначать выражением $a\ket{0}$ + $b\ket{1}$, где $\alpha$ и $\beta$ — комплексные числа, удовлетворяющие условию (\ref{for:ver})

\begin{equation}
\label{for:ver}
|\alpha|^2 + |\beta|^2 = 1. 
\end{equation} 

Измерение кубита мгновенно «схлопывает»  его состояние в базисное – 0 или 1. Вероятности перехода в эти состояние равны соотвественно $|\alpha|^2$ и $|\beta|^2$. 

\subsection{Квантовый регистр}

На кубиты может быть наложена ненаблюдаемая связь -- при всяком изменении над одним из нескольких кубитов остальные меняются согласованно с ним. Таким образом, можно интерпретировать такую совокупность кубитов как заполненный квантовый регистр. Такой регистр может находиться во всех комбинациях составляющих его битов, и, кроме этого, реализовывать завсимости между ними.

\section{Синтез изображения в квантовом представлении}

\subsection{Квантовый пиксельный шейдер}

Пиксельный шейдер -- это программа (которая чаще всего выполняется на графическом процессоре), которая на вход принимает координаты $x$ и $y$ и на выходе выдает цвет пикселя, находящегося в заданных координатах. Для реализации квантового пиксельного шейдера, воспользуемся двумя квантовыми регистрами, каждый из которых состоит из $N$ кубитов. Назовём их $qx$ и $qy$ для осей $x$ и $y$ соотвественно. Таким образом, размер синтезируемого изображения зависит от величины $N$. Например, при $N = 4$, регистры $qx$ и $qy$ будут содержать $2^N = 2^4 = 16$ кубитов, что будет соответствовать размеру изображения 16x16 пикселей. Такой пиксель может быть только черным (1) или белым (0), потому что он будет соответствовать значениям которые может принимать один кубит. 

\subsection{Квантовая фаза}

Произвольное квантовое состояние, обозначаемое $\ket{\psi}$, может быть любая суперпозиция, записанная в виде (\ref{for:psi}):
\begin{equation}
\label{for:psi}
\ket{\psi} = \alpha\ket{0} + \beta\ket{1}
\end{equation}
 базисных векторов. Согласно условию (\ref{for:ver}) и основываясь на том факте, что глобальная фаза не наблюдаема (то есть $\ket{\psi}$ тоже самое что и $e^{j\gamma}\ket{\psi}$) \cite{global-phase}, выражение (\ref{for:psi}) можно переписать в виде (\ref{for:new_ver}):
 
\begin{equation} 
\label{for:new_ver}
\ket{\psi} = \sqrt{1 - p}\ket{0} + e^{j\gamma}\sqrt{p}\ket{1}
\end{equation} где $0 \leq p \leq 1$ -- вероятность того, что бит находится в состоянии 1 и $0 \leq \psi < 2\pi$ -- квантовая фаза.

Применив операцию смены фазы на противоположную (повернуть кубит на 180 градусов), можно изменить состояние кубита аналогично на противоположное. Операцию смены фазы можно применять сразу на несколько кубитов. Таким образом, всего за одну операцию, можно, например, изменить цвет половины (или всех) пикселей синтезируемого изображения. Выигрыш по сравнению с обычными вычислениями очевиден. Кроме того, стало понятно, как можно изобразить прямые. 

\subsection{Усиление комплексной амплитуды}

Допустим, что у нас имеется четырехкубитный квантовый регистр, который содержит одно из трех квантовых состояний, но мы не знаем какое именно. В каждом из этих состояний присутствует некоторое значение с инвертированной фазой. Назовем его помеченным значением. При чтении из квантового регистра, мы получим случайно число с равномерным распределением, и ничего не сможем узнать о том, какое из трех квантовых состояний было исходным.

Введем зеркальную операцию. Её суть заключается в следующих действиях: выполняется инвертирование фазы, берет регистр, находящийся в состоянии $\ket{0}$ и помечает одно из значений регистра. Теперь амплитуды в каждом состоянии очень сильно различаются, и выполнение операции чтения из регистра с большой вероятностью покажет, у какого значения инвертирована фаза -- а следовательно, в каком из трех состояний находился регистр.

\subsection{Квантовая фазовая логика}

Квантовая фазовая логика инвертирует фазу каждого входного значения, которое дает 1 в результате.
Фазовая логика принципиально отличается от любой традиционной логики -- результаты логических операций скрыты в фазах и их невозможно прочитать. Но, при этом, инвертируя фазы в суперпозиции, можно пометить несколько решений в одном регистре. Кроме того, при использовании инвертирования и усиления комплексной амплитуды можно создавать результаты, доступные для чтения.

С помощью комбинации усилении амплитуды и операций фазовой логики, можно сохранить значение логической операции в фазе состояния \cite{PQC}. Таким образом, мы можем описывать более сложные фигуры. 

Окружность задается уравнением вида (\ref{for:circle}):

\begin{equation}
\label{for:circle}
x^2 + y^2 = r^2
\end{equation}

Предположим, что мы хотим заполнить все пиксели, находящиеся внутри окружности, то есть пиксели, подходящие под условие (\ref{for:circle2}):

\begin{equation}
\label{for:circle2}
x^2 + y^2 < r^2
\end{equation}

Для выполнения этого действия нам потребуются выше описанные регистры $qx$ и $qy$, а так же дополнительные регистр-аккумулятор $qacc$. Дальнейший алгоритм таков:

\begin{itemize}
	\item инициализировать регистры $qx$, $qy$ и $qacc$;
	\item ввести регистры $qx$ и $qy$ в суперпозицию;
	\item добавить в регистр $qacc$ сумму квадратов регистров $qx$ и $qy$;
	\item вычесть из регистра $qacc$ квадрат радиуса описываемой окружности;
	\item инвертировать регистр $qacc$ для всех значащих битов;
	\item восстановить регистр $qacc$.
\end{itemize}


\section{Квантовая избыточная выборка}
\section{Колоризация изображения}

\section*{Вывод}

