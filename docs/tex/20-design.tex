\chapter{Конструкторская часть}

\section{Схема квантового компьютера}

Огромный опыт, накопленный в квантовой физике 20 века, дает возможность с оптимизмом смотреть на возможность создания квантового процессора уже в ближайшее время \cite{quantum-computer}. Однако, любой компьютер состоит не только лишь из одного так называемого <<железа>> (hardware), и квантовый компьютер не является исключением. Если задача создания операционной системы для классических вычислений является такой же масштабной задачей, как создание <<железа>>, то в квантовом случае операционная система представляет собой гораздо большую трудность \cite{quantum-computer}.

Схематически квантовый компьютер можно представить в виде структуры, состоящей из трёх блоков:

\begin{itemize}
	\item квантовый процессор;
	\item квантовая операционная система;
	\item пользовательский интерфейс.
\end{itemize}

\subsection{Квантовая коррекция ошибок}
Как уже говорилось в аналитической части, квантовые вычисления выдают только лишь какую-то вероятность значения, а не само значение (это явление называется квантовой декогеренцией). В связи с этим была придумана квантовая коррекция ошибок \cite{quantum-codes}, которая используется для защиты от ошибок из-за квантовой декогеренции. 

\subsection{Пользовательский интерфейс квантового компьютера}

Р. Фейман в своей работе \cite{feynman} предложил пользовательский интерфейс, основанный на массиве квантовых гейтов, которые реализуют простейшие унитарные операторы на малом числе кубитов. Такой интерфейс следует из современного понимания квантовой теории \cite{quantum-computer} и в настоящее время является основным. Важным свойством этого интерфейса является тот факт, что он задействует малое количество кубитов.

К сожалению, такой интерфейс имеет принципиальный недостаток. Этот недостаток состоит в том, что он рассчитан только на моделирование унитарной динамики чистых состояний, что является проблемой для систем с декогерентностью. С декогенертностью частично можно бороться \cite{quantum-codes} -- для этого существуют квантовая коррекция ошибок. Но использование квантовой коррекции ошибок требует дополнительных кубитов, что является проблематичным требованием. Кроме того, к декогерентности оказываются особо чувствительны именно те квантовые состояния, которые возникают при реализации квантовых вычислений \cite{quantum-computer}.

\subsection{Квантовая операционная система}

Квантовая операционная система должна работать в режиме реального времени \cite{quantum-computer}. Это предполагает использование специальных программных примитивов, вместо массива гейтов. Как уже было в главе 2.1.2, основной квантовой интерфейс на данный момент основан на использовании массива квантовых гейтов, что приводит нас к противоречию. 

Ядро квантовой операционной системы является программа, написанная для классического суперкомпьютера. Программные примитивы с высокой точность моделируют динамику малого фрагмента всей модели, а ядро операционной системы управляет динамикой квантового состояния всей системы, причем на большом отрезке времени. На этом моменте мы сталкиваемся со второй проблемой квантовой операционной системы: на данный момент неизвестно, как квантовая теория работает в области сложных (в том числе длительных) процессов. Существует лишь математическая схема квантовых алгоритмов \cite{quantum-computer}.

Таким образом, квантовый компьютер становится лишь экспериментальным устройством, на создание которого уйдет ещё какое-то время. В настоящее время мы можем лишь эмулировать квантовые вычисления с использованием обычных компьютеров.

\section{Эмуляция квантовых вычислений}

Квантовые вычисления можно эмулировать на обычном компьютере:

\begin{itemize}
	\item состояние кубита можно представить комплексным числом, занимающим от $2N$ бит, где $N$ -- разрядность процессора;
	\item состояние из $K$ связанных кубитов можно представить в виде $2^K$ комплексных чисел;
	\item квантовую операцию над $M$ кубитами можно представить матрицей $2^M \times 2^M$.
\end{itemize}

В таком случае, для хранения эмулируемых состояний:

\begin{itemize}
	\item 10 кубитов нужны 8 Кб;
	\item 20 кубитов нужны 8 Мб;
	\item 30 кубитов нужны 8 Гб;
	\item и так далее.
\end{itemize}

Таким образом, предел симуляции квантового компьютера на классических компьютерах обусловлен количеством оперативной памяти \cite{habr-quantum-computers}.

Существует много различных реализаций квантовых вычислений \cite{habr-quantum-computers}, но все они основаны на одних и тех же принципах:

\begin{itemize}
	\item используется некоторая схема хранения всех состояний кубитов;
	\item все операции над кубитами обеспечиваются с помощью специальных унитарных матриц;
	\item используется рандомизатор для внесения неопределенности в квантовой системе;
\end{itemize}

\subsection{Квантовое превосходство}

Квантовое превосходство -- способность квантовых вычислительных устройств решать проблемы, которые классические компьютеры практически не могут решить \cite{quantum-supremacy}. Достижение квантового превосходства означает, что задачи, такие как, например, факторизацию больших чисел можно решать за адекватное время. Кроме этого, квантовый компьютер с запущенной на нем некоторой квантовой схеме, то результат этой работы будет невозможно сэмулировать на обычном компьютере, то есть классический компьютер воссоздать результат работы такой схемы будет не в состоянии.

Подводя итог, можно прийти к выводу, что для эмуляции квантовых вычислений в выбранном мною алгоритме можно будет воспользоваться лишь не более чем 30 -- 32 кубитами.

\section{Алгоритм квантовой избыточной выборки}

В общем виде псевдокод алгоритма квантовой избыточной выборки можно описать следующим образом:

\begin{enumerate}
	\item для всех пикселей на холсте -- перевести в состояние 1 (черный цвет);
	\item инициализировать квантовый регистр-счетчик, хранящий количество итераций усиления квантовой амплитуды;
	\item инициализировать квантовую поисковую таблицу и карту достоверности пустыми значениями;
	\item сформировать и заполнить квантовую поисковую таблицу;
	\item посчитать количество инвертируемых пикселей (пользуясь поисковой таблицей);
	\item использовать эту величину для определения яркости пикселя;
	\item заполнить соответствующую ячейку таблицы достоверности;
	\item повторить пункты 5 - 7 для всех пикселей холста;
	\item повторить пункты 1 - 8 для <<красного>> и <<синего>> слоя (либо слоев состоящих из любых други цветов) (см. аналитическую часть);
	\item объединить полученные изображения в единое цветное изображение.
\end{enumerate}

\subsection{Формирование квантовой поисковой таблицы}

Ниже представлен алгоритм формирования поисковой таблицы. Алгоритм следует выполнить в виде подпрограммы.

\begin{enumerate}
	\item инициализировать и ввести в суперпозицию вспомогательные регистры $qcount$ и $qxy$;
	\item выполнить усиление квантовой амплитуды на регистр $qxy$ столько раз, из скольких битов состоит регистр $qcount$;
	\item применить квантовое преобразование Фурье на регистр qcount;
	\item прочитать значение каждого из битов qcount и заполнить ими ячейки таблицы.
\end{enumerate}

\subsection{Формирование квантовой карты достоверности}

\begin{enumerate}
	\item считать все значения из таблицы поиска в заданной строке;
	\item поделить сумму этих значений на длину строки;
	\item на основе полученного посчитать вероятность ошибки в заданном пикселе и занести его в соответствующую ячейку карты достоверности.
\end{enumerate}

\section*{Вывод}

В данном разделе было рассмотренно устройство реалистического квантового компьютера и был сделан вывод, что на данный момент это лишь гипотетическое устройство. Была рассмотренна схема эмуляция квантовых вычислений на обычных компьютерах и на основе этой информации описан алгоритм квантовой супер выборки с учетом выявленных и описанных ограничений эмуляции таких вычислений.
