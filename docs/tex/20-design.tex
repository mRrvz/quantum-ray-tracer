\chapter{Конструкторская часть}

\section{Схема квантового компьютера}

Опыт, накопленный в квантовой физике 20 века, дает надежду на то, что <<железо>> (англ. hardware) для квантового компьютера возможно реализовать в недалеком будущем \cite{quantum-computer}. Однако, любой компьютер состоит не только лишь из одного так называемого <<железа>>, и квантовый компьютер не является исключением. Если задача создания операционной системы для классических вычислений является такой же масштабной задачей, как создание <<железа>>, то в квантовом случае операционная система представляет собой гораздо большую трудность \cite{quantum-computer}.

Схематически квантовый компьютер можно представить в виде структуры, состоящей из трёх блоков:

\begin{itemize}
	\item квантовый процессор;
	\item квантовая операционная система;
	\item пользовательский интерфейс.
\end{itemize}

\subsection{Квантовая коррекция ошибок}
Как уже говорилось в аналитической части, квантовые вычисления выдают только лишь какую-то вероятность значения, а не само значение (это явление называется квантовой декогеренцией \cite{decog}). В связи с этим была придумана квантовая коррекция ошибок \cite{quantum-codes}, которая используется для защиты от ошибок из-за квантовой декогеренции.

\subsection{Пользовательский интерфейс квантового компьютера}\label{qgui}

Р. Фейман в своей работе \cite{feynman} предложил пользовательский интерфейс, основанный на массиве квантовых гейтов, которые реализуют простейшие унитарные операторы на малом числе кубитов. Такой интерфейс следует из современного понимания квантовой теории \cite{quantum-computer} и в настоящее время является основным. Важным свойством этого интерфейса является тот факт, что он задействует малое количество кубитов.

К сожалению, такой интерфейс подходит для систем с декогерентностью. С декогенертностью частично можно бороться \cite{quantum-codes} -- для этого существуют квантовая коррекция ошибок. Но использование квантовой коррекции ошибок требует дополнительных кубитов, что является проблематичным требованием. Кроме того, к декогерентности оказываются особо чувствительны именно те квантовые состояния, которые возникают при реализации квантовых вычислений \cite{quantum-computer}.

\subsection{Квантовая операционная система}

Квантовая операционная система должна работать в режиме реального времени \cite{quantum-computer}. Это предполагает использование специальных программных примитивов, вместо массива гейтов. Как уже было указано в главе \ref{qgui}, основной квантовой интерфейс на данный момент основан на использовании массива квантовых гейтов, что приводит нас к противоречию. 

Ядро квантовой операционной системы является программа, написанная для классического суперкомпьютера. Программные примитивы с высокой точность моделируют динамику малого фрагмента всей модели, а ядро операционной системы управляет динамикой квантового состояния всей системы, причем на большом отрезке времени. На этом моменте мы сталкиваемся со второй проблемой квантовой операционной системы: на данный момент неизвестно, как квантовая теория работает в области сложных (в том числе длительных) процессов. Существует лишь математическая схема квантовых алгоритмов \cite{quantum-computer}.

Таким образом, квантовый компьютер становится лишь экспериментальным устройством, на создание которого уйдет ещё какое-то время. В настоящее время мы можем лишь эмулировать квантовые вычисления с использованием обычных компьютеров.

\section{Эмуляция квантовых вычислений}\label{emulating}

Квантовые вычисления возможно эмулировать на обычном компьютере \cite{qmemory}:

\begin{itemize}
	\item состояние кубита можно представить комплексным числом, занимающим от $2N$ бит, где $N$ -- разрядность процессора;
	\item состояние из $K$ связанных кубитов можно представить в виде $2^K$ комплексных чисел;
	\item квантовую операцию над $M$ кубитами можно представить матрицей $2^M \times 2^M$.
\end{itemize}

В таком случае, для хранения эмулируемых состояний:

\begin{itemize}
	\item 10 кубитов нужны 8 Кб;
	\item 20 кубитов нужны 8 Мб;
	\item 30 кубитов нужны 8 Гб;
	\item и так далее.
\end{itemize}

Таким образом, предел симуляции квантового компьютера на классических компьютерах обусловлен количеством оперативной памяти \cite{habr-quantum-computers}. Можно сделать вывод, что при эмуляции квантовых вычислений в выбранном мною алгоритме можно будет воспользоваться лишь не более чем 30 -- 32 кубитами.

Существует много различных реализаций квантовых вычислений \cite{habr-quantum-computers}, но все они основаны на одних и тех же принципах:

\begin{itemize}
	\item используется некоторая схема хранения всех состояний кубитов;
	\item все операции над кубитами обеспечиваются с помощью специальных унитарных матриц \cite{umatrix};
	\item используется рандомизатор для внесения неопределенности в квантовой системе;
\end{itemize}

\subsection{Квантовое превосходство}

Квантовое превосходство -- способность квантовых вычислительных устройств решать проблемы, которые классические компьютеры практически не могут решить \cite{quantum-supremacy}. Достижение квантового превосходства означает, что задачи, такие как, например, факторизацию больших чисел можно решать за адекватное время. Кроме этого, квантовый компьютер с запущенной на нем некоторой квантовой схеме, то результат этой работы будет невозможно сэмулировать на обычном компьютере, то есть классический компьютер воссоздать результат работы такой схемы будет не в состоянии.

\section{Структуры данных для квантового алгоритма избыточной выборки}

\subsection{Квантовая поисковая таблица}

Квантовую поисковую таблица можно реализовать в виде обычной матрицы. Такая реализация проста и вытекает из главы \ref{table-q}. В строках матрицы будут хранится результаты чтения значения из квантового регистра, а в столбцах перечислены возможные количества субпикселей в квантовом шейдере, которые могут привести к такому значению, который был получен путём чтения квантового регистра. 

Во время синтеза изображения, в строках матрицы будет постоянно проводиться поиск нужного значения. Особенно, если данная квантовая таблица будет использоваться несколько. Для ускорения поиска, после заполнения строк матрицы, можно отсортировать их в и дальнейшем пользоваться бинарным поиском \cite{binary-search}, что ускорит итоговое время работы алгоритма.

\subsection{Квантовая карта достоверности}

Квантовую карту достоверности реализуют в виде матрицы, так как очень легко провести параллель этой матрицы на синтезируемое изображение. Для каждой ячейки матрицы будет описывать пиксель в итоговом изображении. Номер строки соответствует координате $x$ в итоговом изображение, номер столбца, соотвественно, координату $y$. Сама ячейка матрицы хранит вероятность, что цвет данного пикселя был определен правильно.

\section{Алгоритм квантовой избыточной выборки}

В общем виде псевдокод алгоритма квантовой избыточной выборки (\ref{alg:qss}) можно описать следующим образом:

\begin{algorithm}
	\caption{Квантовая избыточная выборка}
	\label{alg:qss}
	\begin{algorithmic}[1]
		\ForAll {цвета колоризации}
			\ForAll {пиксели холста}
				\State перевести пиксель в состояние 1
			\EndFor
			\State инициализировать регистр, хранящий количество итераций усиления квантовой амплитуды
			\State инициализировать поисковую таблицу и карту достоверности
			\State сформировать и заполнить квантовую поисковую таблицу
			\ForAll {пиксели холста}
				\State посчитать количество инвертируемых пикселей 
				\State использовать эту величину для определения яркости пикселя
				\State заполнить соответствующую ячейку карты достоверности
			\EndFor
		\EndFor
		\State объединить полученные изображения в единое цветное изображение
	\end{algorithmic}
\end{algorithm}

\subsection{Формирование квантовой поисковой таблицы}

Ниже представлен алгоритм (\ref{alg:table}) формирования поисковой таблицы. Алгоритм следует оформить в виде подпрограммы.

\begin{algorithm}
	\caption{Формирование квантовой поисковой таблицы}
	\label{alg:table}
	\begin{algorithmic}[1]
		\State инициализировать и ввести в суперпозицию вспомогательные регистры $qcount$ и $qxy$
		\State выполнить усиление квантовой амплитуды на регистр $qxy$ столько раз, из скольких битов состоит регистр $qcount$
		\State применить квантовое преобразование Фурье на регистр $qcount$
		\State прочитать значение каждого из битов $qcount$ и заполнить ими ячейки таблицы
	\end{algorithmic}
\end{algorithm}

\subsection{Формирование квантовой карты достоверности}

В алгоритме (\ref{alg:map}) представлен псевдокод формирования квантовой карты достоверности.

\begin{algorithm}
	\caption{Формирование квантовой карты достоверности}
	\label{alg:map}
	\begin{algorithmic}[1]
		\State считать все значения из таблицы поиска в заданной строке
		\State поделить сумму этих значений на длину строки
		\State на основе полученного посчитать вероятность ошибки в заданном пикселе и занести его в соответствующую ячейку карты достоверности
	\end{algorithmic}
\end{algorithm}


\section*{Вывод}

В данном разделе было рассмотренно устройств квантового компьютера и был сделан вывод, что на данный момент это лишь гипотетическое устройство. Была рассмотренна схема эмуляция квантовых вычислений на обычных компьютерах и на основе этой информации описан алгоритм квантовой супер выборки с учетом выявленных и описанных ограничений эмуляции таких вычислений. Кроме того, были описаны структуры данных, которые понадобятся для реализации данного алгоритма.
