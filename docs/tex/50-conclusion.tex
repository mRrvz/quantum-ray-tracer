\chapter*{Заключение}

Во время выполнения курсового проекта было реализованно программное обеспечение в котором были реализованны квантовые алгоритмы, которые в возможно применить в алгоритме трассировки лучей с целью его улучшения.

Были проанализированы и рассмотренны существующие алгоритмы трассировки лучей. В качестве той части алгоритма, которую стоит заменить на квантовую, был выбран алгоритм избыточной выборки. Данный выбор был сделан на основе анализа основных принципов построения квантовых вычислений, которые позволили выбрать наиболее подходящее решение для поставленной задачи.

В ходе выполнения поставленной задачи были получены знания в области компьютерной графики, а так же в области квантовых вычислений. Были изучены принципы эмуляции квантовых вычислений на обычных компьютерах. Поиск подходящего решения для поставленной задачи позволил повысить навыки и анализа информации.

В результате проведенной работы было получено программное обеспечение, доказывающее применимость квантовых алгоритмов в области компьютерной графики. Разработанный программный продукт реализует квантовый алгоритм избыточной выборки, который в дальнейшем возможно применить для реализации алгоритма трассировки лучей.

В ходе выполнения эксперементально-исследовательской части было установленно, что уровень шума классического алгоримта и квантового алгоритма избыточной выборки примерно одинаков. Было выявленно, что полученные изображения отличаются в характере шума: квантовый алгоритм генерирует в среднем на 15\% больше идеальных пикселей чем его классический аналог. При этом, остальные пиксели крайне зашумленны, что может ускорить постобработку изображения, например удаление и замену этих пикселей.

\addcontentsline{toc}{chapter}{Заключение}
