\chapter*{Введение}
\addcontentsline{toc}{chapter}{Введение}

Трассировка лучей -- метод геометрической оптики -- исследование оптических систeм путём отслеживания взаимодействия отдельных лучей с поверхностями. Специальный алгоритм отслеживает путь луча, начиная от объекта освещения до объектов, расположенных на сцене. Далее, алгоритм создает симуляцию взаимодействия с объектами: отражение, преломление и так далее. Полученная информация используются для определения цвета каждого пикселя в итоговом изображении. 

Качество результирующего изображения, а также стоимость вычислений возрастают с увеличением количества испускаемых лучей. Трассировка лучей коммерчески используется в киноиндустрии \cite{path-traced-movies}, что мотивировано снижением затрат. Для высокобюджетного фильма с 24 кадрами в секунду, на рендеринг одного кадра уходит до двух часов \cite{path-traced-movies}. Сокращение этого времени хотя бы на 50\% приведет к деловой мотивации \cite{path-traced-movies}. 

Квантовые вычисления -- это альтернатива классическим алгоритмам, основанная на процессах квантовой физики, которая гласит, что без взаимодействия с другими частицами (то есть до момента измерения), электрон не
размещен в однозначных координатах, а одновременно расположен
в каждой точке орбиты. Область, в которой расположен электрон, называется электронным облаком. В ходе опыта Юнга, эксперимента с двумя щелями один электрон проходит одновременно через обе щели, интерферируя при этом с самим собой. Только при измерении эта неопределенность схлопывается и координаты электрона становятся однозначными.

В квантовых вычислениях физические свойства квантовых объектов реализованы в кубитах. Классический бит принимает только два значения – 0 или 1. Кубит до измерения принимает
одновременно оба значения. Из-за этого, кубит принято обозначать выражением $a\ket{0}$ + $b\ket{1}$, где $A$ и $B$ — комплексные числа, удовлетворяющие условию $|A|^2 + |B|^2 = 1$. Измерение кубита мгновенно «схлопывает»  его состояние в базисное – 0 или 1. При этом <<облако>> коллапсирует
в точку, первоначальное состояние разрушается, и информация безвозвратно теряется. Это свойство, например, применяется в генeраторе истиннo случайных чисел \cite{generator}. %Кубит вводится в такое состояние, при котором результатом измерения могут быть 1 или 0 с одинаковой вероятностью. Это состояние описывается так: $\frac{1}{\sqrt{2} \ket{0}} + \frac{1}{\sqrt{2} \ket{1}}$

Случаи квантового ускорения, на фоне массы классических алгоритмов, редки \cite{quantum-computers-speed-up}. Однако, это не умаляет значения квантовых вычислений, потому что они способны ускорить выполнение задач переборного типа. 

Цель работы -- реализовать ПО, которое
использует квантовый алгоритм трассировки лучей.

Чтобы достигнуть поставленной цели, требуется решить следующие задачи:

\begin{itemize}
    \item проанализировать стандартный алгоритм трассировки лучей, чтобы
понять какую часть вычислений стоит на квантовые;
    \item выбрать структуру геометрической модели сцены;
    \item проанализировать и выбрать квантовые алгоритмы, которые будут использоваться в алгоритме трассировки;
    \item доказать построенную концепцию, реализовав квантовый алгоритм трассировки лучей.
\end{itemize}
