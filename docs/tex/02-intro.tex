\chapter*{Введение}
\addcontentsline{toc}{chapter}{Введение}

Трассировка лучей -- метод геометрической оптики -- исследование оптических систeм путём отслеживания взаимодействия отдельных лучей с поверхностями. Специальный алгоритм отслеживает путь луча, начиная от объекта освещения до объектов, расположенных на сцене. Далее, алгоритм создает симуляцию взаимодействия с объектами: отражение, преломление и так далее. Полученная информация используется для определения цвета каждого пикселя в итоговом изображении. 

Качество результирующего изображения на прямую зависит от количества испускаемых лучей, а увеличение количества лучей, в свою очередь, требует повышения затрат вычислительных ресурсов. Трассировка лучей используется в киноиндустрии \cite{path-traced-movies}, что мотивирует искать способы снижения снижения затрат на синтез изображения с помощью трассировки лучей. Так, например, для высокобюджетного фильма с 24 кадрами в секунду, на рендеринг одного кадра уходит до двух часов \cite{path-traced-movies}.

Квантовые вычисления -- это альтернатива классическим алгоритмам, основанная на процессах квантовой физики, которая гласит, что без взаимодействия с другими частицами (то есть до момента измерения), электрон не размещен в однозначных координатах, а одновременно расположен в каждой точке орбиты. Область, в которой расположен электрон, называется электронным облаком, а феномен нахождения в каждой точке орбиты -- суперпозиция. В ходе опыта Юнга, эксперимента с двумя щелями один электрон проходит одновременно через обе щели, интерферируя при этом с самим собой. Только при измерении эта неопределенность схлопывается и координаты электрона становятся однозначными.
 %Кубит вводится в такое состояние, при котором результатом измерения могут быть 1 или 0 с одинаковой вероятностью. Это состояние описывается так: $\frac{1}{\sqrt{2} \ket{0}} + \frac{1}{\sqrt{2} \ket{1}}$

Случаи, когда квантовый алгоритм работает хотя бы немного быстрее чем его классическая версия, редки \cite{quantum-computers-speed-up}. Однако, это не умаляет значения квантовых вычислений, потому что они способны ускорить выполнение задач переборного типа. 

Цель работы -- реализовать ПО, в котором реализованны квантовые алгоритмы, которые в дальнейшем возможно применить в алгоритме трассировки лучей с целью его улучшения.

Чтобы достигнуть поставленной цели, требуется решить следующие задачи:

\begin{itemize}
    \item проанализировать алгоритм трассировки лучей, чтобы понять какую часть вычислений стоит заменить на квантовые;
    \item проанализировать и сконструировать выбрать квантовые алгоритмы и структуры данных, которые возможно использовать в алгоритме трассировки;
    \item реализовать выбранные квантовые алгоритмы.
    \item провести сравнение рассматриваемых алгоритмов с использованием квантовых и традиционных вычислений.
\end{itemize}
